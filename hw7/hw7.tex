\documentclass[12pt]{article}
\usepackage{graphicx}
\usepackage{geometry}
\usepackage{float}
\usepackage{amssymb}
\usepackage{amsmath}
\usepackage{pythonhighlight}

\geometry{left=2.0cm,right=2.0cm,top=2.5cm,bottom=2.5cm}

\title{
    \textbf{\Huge ECE374 Fall2020} \\
    \textbf{\Large Homework7}
    }

\author{\large Name: Zhang Yichi 3180111309}

\date{Nov. $15^{th}$ 2020}

\begin{document}
\setlength{\parindent}{0pt}
\maketitle

\section*{11.1-1}
Traverse the direct-address table T, compare each one of the element and find the largest one. The worst-case time complexity is O(m) where m is the length of T.

\section*{11.1-2}
\inputpython{hw7.py}{31}{47}
The i-th slot corresponds to the key i. If the bit in the i-th slot is 1, it means that there is an element i, 0 otherwise. As the codes shown above, the dictionary
operations run in O(1) time.

\section*{11.2-1}
let X denotes collide or not, Y denotes the total number of collision. Therefore, $Y=\sum_{k\not=l}X$ and
$
    E[Y]=E[\sum_{k\not=l}X]
    =\sum_{k\not=l}E[X]
    =\sum_{k\not=l}\dfrac{1}{m}
    =\binom{n}{2}\dfrac{1}{m}
    =\dfrac{n(n-1)}{2}\dfrac{1}{m}
    =\dfrac{n(n-1)}{2m}
$

\section*{11.2-3}
For insertions and deletions, the time complexity does not change, still $\Theta(1+\alpha)$, one for computing $h(k)$, $\alpha$ for rearrange the items after. For successful and unsuccessful searches, the time complexity becomes $\Theta(1+lg(\alpha))$ because the time complexity for searching in a sorted list is $O(lg(n))$.

\section*{11.4-1}
\inputpython{hw7.py}{49}{87}
$h'(k)=k$, $m=11$, $c_1=1$, $c_2=3$, $h_1(k)=k$, $h_2(k)=1+(k\mod{(m-1)})$\\
Insert 10, 22, 31, 4, 15, 28, 17, 88, 59.\\

Linear probing: $h(k,i)=(h'(k)+i)\mod{m}=(k+i)\mod{11}$\\
The result is: [22, 88, None, None, 4, 15, 28, 17, 59, 31, 10]\\

Quadratic probing: $h(k,i)=(h'(k)+c_1i+c_2i^2)\mod{m}=(k+i+3i^2)\mod{11}$\\
The result is: [22, None, 88, 17, 4, None, 28, 59, 15, 31, 10]\\

Double hashing: $h(k,i)=(h_1(k)+ih_2(k))\mod{m}=(k+i(1+(k\mod{10})))\mod{11}$\\
The result is: [22, None, 59, 17, 4, 15, 28, 88, None, 31, 10]

\section*{11.4-2}
\begin{python}
Hash-Deletion(T,k)
i = 0
repeat
    j = h(k,i)
    if T[j] == k
        T[j] = DELETED
        return j
    else i = i + 1
until i == m or T[j] == NIL
error "no k found"
\end{python}
\begin{python}
Modified-Hash-Insert(T,k)
i = 0
repeat
    j = h(k,i)
    if T[j] == NIL or T[j] == DELETED
        T[j] = k
        return j
    else i = i + 1
until i == m
error "hash table overflow"
\end{python}
\end{document}