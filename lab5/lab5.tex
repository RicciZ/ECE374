\documentclass[12pt]{article}
\usepackage{graphicx}
\usepackage{geometry}
\usepackage{float}
\usepackage{pythonhighlight}
\usepackage{amsmath}

% \geometry{left=2.0cm,right=2.0cm,top=2.5cm,bottom=2.5cm}

\title{
    \textbf{\Huge ECE374 Fall2020} \\
    \textbf{\Large Lab5: Counting Sort}
    }

\author{\large Name: Zhang Yichi 3180111309}

\date{Nov. $2^{nd}$ 2020}

\begin{document}
\setlength{\parindent}{0pt}
\maketitle
\pagestyle{empty}
\section{Introduction}
In this lab, we will implement counting sort that accepts negative numbers. Counting sort is a sorting algorithm with time complexity O(k+n).

\section{Python Code for Counting Sort}
\inputpython{lab5.py}{3}{17}
minx is the minimum number in A. \\
maxx is the maximum number in A. \\
k is the number of buckets needed. I use maxx-minx+1 to normalize these numbers and minimize the buckets needed. \\
First, create k empty buckets, C. Then, go through A, put the numbers in A into buckets in C and record the numbers of the numbers in each buckets. Finally, take the numbers out in order and append them to B. B is the sorted array.

\section{Test Example for Counting Sort}
\inputpython{lab5.py}{19}{22}
\begin{figure}[H]
    \centering
    \includegraphics[width=14cm]{test.png}
    \caption{counting sort test result}
\end{figure}
Here I generate 20 random numbers from -10 to 10, sort them with the counting\_sort function I developed and output them to see the results.

\section{Time Complexity Analysis}
The first loop is O(k). The second loop is O(n). The third loop consists of two loops. The outer loop is O(k). The inner loop iterates C[i]. C contains the number of numbers from A so the sum of C[i] is n. Thus, the inner loop iterates n times in total. Therefore, the third loop is O(k+n). In all, the time complexity of this counting sort is O(k+n). 

\end{document}