\documentclass[12pt]{article}
\usepackage{graphicx}
\usepackage{geometry}
\usepackage{float}
\usepackage{pythonhighlight}
\usepackage{amsmath}

\geometry{left=2.0cm,right=2.0cm,top=2.5cm,bottom=2.5cm}

\title{
    \textbf{\Huge ECE374 Fall2020} \\
    \textbf{\Large Lab10: Longest Common Subsequence}
    }

\author{\large Name: Zhang Yichi 3180111309}

\date{Dec. $15^{th}$ 2020}

\begin{document}
\setlength{\parindent}{0pt}
\maketitle
\pagestyle{empty}
\section{Introduction}
In this lab, we will implement a program to search for the longest common subsequence from two strings using brute force and dynamic programming algorithm.

\section{Python Code for the LCS Problem}
\inputpython{lab10.py}{1}{54}
There are two main functions. LCS\_bf solves the LCS problem via brute force and LCS\_dp solves it by dynamic programming. For LCS\_bf, I generate all the subsequences, check if they are the subsequences for both two input strings X, Y and compare them to find the longest one. While for LCS\_dp, I first build a 2D table where each entry [i,j] stores the information for the LCS of X[1...i] and Y[1...j] and then complete it from bottom up by two for loops.

\section{Test Example for the LCS Problem}
\inputpython{lab10.py}{56}{59}
\begin{figure}[H]
    \centering
    \includegraphics[width=12cm]{test.png}
    \caption{LCS problem test results.}
\end{figure}
Call the two functions to see the results.

\section{Time Complexity}
Brute force: $O(n2^m)$ \\
Dynamic programming: $O(mn)$


\end{document}