\documentclass[12pt]{article}
\usepackage{graphicx}
\usepackage{geometry}
\usepackage{float}
\usepackage{pythonhighlight}
\usepackage{amsmath}

\geometry{left=2.0cm,right=2.0cm,top=2.5cm,bottom=2.5cm}

\title{
    \textbf{\Huge ECE374 Fall2020} \\
    \textbf{\Large Lab7: Hash Table Insertion}
    }

\author{\large Name: Zhang Yichi 3180111309}

\date{Nov. $16^{th}$ 2020}

\begin{document}
\setlength{\parindent}{0pt}
\maketitle
\pagestyle{empty}
\section{Introduction}
In this lab, we will implement and analyze the hash algorithm with quadratic probing and double hashing.

\section{Python Code for Hash}
\inputpython{lab7.py}{1}{19}
Each function implements a hash algorithm. We take $h'(k)=k$, $m=21$, $c_1=1$, $c_2=3$, $h_1(k)=k$, $h_2(k)=1+(k\mod{(m-1)})$. The first one is hash with quadratic probing $h(k,i)=(h'(k)+c_1i+c_2i^2)\mod{m}=(k+i+3i^2)\mod{11}$ and the second one is hash with double hashing $h(k,i)=(h_1(k)+ih_2(k))\mod{m}=(k+i(1+(k\mod{20})))\mod{21}$.


\section{Test Example for Hash}
\inputpython{lab7.py}{21}{32}
\begin{figure}[H]
    \centering
    \includegraphics[width=16cm]{test.png}
    \caption{hash insert test result}
\end{figure}

\section{Time Complexity Analysis}
Both the insertion with quadratic probing and double hashing have the time complexity of $\Theta(1+\alpha)$ where $\alpha=\frac{n}{m}$.


\end{document}